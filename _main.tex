% Options for packages loaded elsewhere
\PassOptionsToPackage{unicode}{hyperref}
\PassOptionsToPackage{hyphens}{url}
%
\documentclass[
  openany]{book}
\usepackage{lmodern}
\usepackage{amssymb,amsmath}
\usepackage{ifxetex,ifluatex}
\ifnum 0\ifxetex 1\fi\ifluatex 1\fi=0 % if pdftex
  \usepackage[T1]{fontenc}
  \usepackage[utf8]{inputenc}
  \usepackage{textcomp} % provide euro and other symbols
\else % if luatex or xetex
  \usepackage{unicode-math}
  \defaultfontfeatures{Scale=MatchLowercase}
  \defaultfontfeatures[\rmfamily]{Ligatures=TeX,Scale=1}
\fi
% Use upquote if available, for straight quotes in verbatim environments
\IfFileExists{upquote.sty}{\usepackage{upquote}}{}
\IfFileExists{microtype.sty}{% use microtype if available
  \usepackage[]{microtype}
  \UseMicrotypeSet[protrusion]{basicmath} % disable protrusion for tt fonts
}{}
\makeatletter
\@ifundefined{KOMAClassName}{% if non-KOMA class
  \IfFileExists{parskip.sty}{%
    \usepackage{parskip}
  }{% else
    \setlength{\parindent}{0pt}
    \setlength{\parskip}{6pt plus 2pt minus 1pt}}
}{% if KOMA class
  \KOMAoptions{parskip=half}}
\makeatother
\usepackage{xcolor}
\IfFileExists{xurl.sty}{\usepackage{xurl}}{} % add URL line breaks if available
\IfFileExists{bookmark.sty}{\usepackage{bookmark}}{\usepackage{hyperref}}
\hypersetup{
  pdftitle={Wellcome Historical Medical Museum Flimsy Slips: Guide for Transcribers},
  pdfauthor={Alexandra Eveleigh},
  hidelinks,
  pdfcreator={LaTeX via pandoc}}
\urlstyle{same} % disable monospaced font for URLs
\usepackage{longtable,booktabs}
% Correct order of tables after \paragraph or \subparagraph
\usepackage{etoolbox}
\makeatletter
\patchcmd\longtable{\par}{\if@noskipsec\mbox{}\fi\par}{}{}
\makeatother
% Allow footnotes in longtable head/foot
\IfFileExists{footnotehyper.sty}{\usepackage{footnotehyper}}{\usepackage{footnote}}
\makesavenoteenv{longtable}
\usepackage{graphicx,grffile}
\makeatletter
\def\maxwidth{\ifdim\Gin@nat@width>\linewidth\linewidth\else\Gin@nat@width\fi}
\def\maxheight{\ifdim\Gin@nat@height>\textheight\textheight\else\Gin@nat@height\fi}
\makeatother
% Scale images if necessary, so that they will not overflow the page
% margins by default, and it is still possible to overwrite the defaults
% using explicit options in \includegraphics[width, height, ...]{}
\setkeys{Gin}{width=\maxwidth,height=\maxheight,keepaspectratio}
% Set default figure placement to htbp
\makeatletter
\def\fps@figure{htbp}
\makeatother
\setlength{\emergencystretch}{3em} % prevent overfull lines
\providecommand{\tightlist}{%
  \setlength{\itemsep}{0pt}\setlength{\parskip}{0pt}}
\setcounter{secnumdepth}{5}

\title{Wellcome Historical Medical Museum Flimsy Slips: Guide for Transcribers}
\author{Alexandra Eveleigh}
\date{}

\begin{document}
\maketitle

{
\setcounter{tocdepth}{1}
\tableofcontents
}
07/10/2020
Version 4.0

\hypertarget{transcription-setup}{%
\chapter{Transcription Setup}\label{transcription-setup}}

You can transcribe either directly from the \href{https://wellcomecollection.org}{Wellcome Collection} website or from a downloaded PDF of the digitised images.

Each transcription volunteer will be sent a transcription spreadsheet template and a URL link to a catalogue entry
e.g.~\url{https://wellcomecollection.org/works/z5p7gm97}

Click on the `View' button on the catalogue entry, and this takes you to a specific set of images corresponding to a box full of `flimsy slips' - index cards printed on lightweight, `flimsy' paper. If a message about `Archival material less than 100 years old' pops up, click `Accept Terms and Open' in order to load the flimsy slips into the Viewer.

\hypertarget{transcribing-directly-from-the-wellcome-collection-website}{%
\section{Transcribing directly from the Wellcome Collection website}\label{transcribing-directly-from-the-wellcome-collection-website}}

The images on the Wellcome Collection website are higher quality than those available when you download a PDF. This means that you can expand the image to full screen without loss of resolution using the zoom controls built into the Viewer.

\textbf{Image Number} If you decide to transcribe directly from the website, the image number is the number at the top middle of the Viewer window to the left of the `/'.

\hypertarget{using-grp_wellcome_transcribers-on-teams}{%
\chapter{Using Grp\_Wellcome\_Transcribers on Teams}\label{using-grp_wellcome_transcribers-on-teams}}

As part of the Transcribe Wellcome team you will be invited to join the Transcribers' group on Microsoft Teams. The group is an important part of the transcription process and is the best way to keep up to date with updates, advice and events.

There are several channels within the group:

\begin{itemize}
\item
  \textbf{General:} Where the organisation team/admin post important updates and event announcements. The `Files' section on this channel is where members upload spreadsheets when they want feedback or once they have completed their transcriptions.
\item
  \textbf{Library Accession Cards:} A channel where members can ask for and read advice in transcribing the Library Accession Registers. Members should also use this space to highlight any interesting finds from the Registers and inform the admin team when they have submitted completed Library spreadsheets.
\item
  \textbf{Museum Index Cards:} A channel for members to ask for and read advice in transcribing the Museum Flimsy Slips. Members should also use this space to highlight any interesting finds from the Slips and inform the admin team when they have submitted completed Museum spreadsheets.
\item
  \textbf{Visual Material Registers:} A channel where members can ask for and read advice in transcribing the Visual Material Registers. Members should also use this space to highlight any interesting finds from the Registers and inform the admin team when they have submitted completed Visual Material spreadsheets.
\item
  \textbf{Transcribers:} A chat channel for members to introduce themselves and ask for Library or Museum spreadsheets.
\item
  \textbf{FAQ:} A channel answering frequently asked questions on all aspects of the transcription work.
\item
  \textbf{Transcribe Wellcome {[}TW{]}:} A channel for updates, questions and requests regarding the Transcribe Wellcome database.
\end{itemize}

\#General Transcription Guidelines

\#\#The Transcription Spreadsheet

The column headings in the transcription spreadsheet correspond -- broadly -- to the headings printed on the flimsy slips, plus some additional columns for recording details which may be useful for future data analysis and online search.

The style of the printed flimsy slips changed over time, and according to the purpose for which they were used by Wellcome Historical Medical Museum, and some headings were rarely used.

\#\#General instructions

Generally, each flimsy slip will be transcribed onto one row of the spreadsheet, except:

\begin{itemize}
\tightlist
\item
  When one slip has more than one object described in detail, in which case split the description over separate rows, with one row for each object. Give each row/description a sub number in column E {[}1, 2, 3, 4 etc.{]} or an alphabetical identifier if letters of the alphabet have been used to distinguish items on the flimsy slip itself (this is rare, but there are some known examples).
\item
  When one slip represents a short running sequence of A numbers, in which case use a separate row for each number in the sequence, transcribing the details from the card onto each row. Note that you will not be able to copy and paste an entire row of the spreadsheet (which should prevent you accidentally duplicating rows for the same A number), but you can copy and paste column D across to column V if the details are the same as the preceding row.
\item
  Some flimsy slips will have pencil written in addition to pen. If you see this, please transcribe as per usual in their respective fields i.e if half the description is in pen and the other in pencil, transcriber all of it in the description (column K).
\item
  If it is unclear where the pencil writing is meant to go, please enter it into additional notes (column Q) and highlight this in the transcribers note column.
\item
  If there is an original question mark within a field and/or writing on your flimsy slip, please transcribe this exactly as it shows and make a note that it is in the original flimsy slip in the transcribers notes column.
\item
  If there is something you are unsure of, please note it as {[}\ldots{]} and make sure there is an `unsure' note in the transcribers column, describing what it is you are unsure of.
\item
  For any text which has been scored through, do the same in the transcription. If you do use the strikethrough, or any other formatted text, you must use the appropriate tagging e.g.~{[}transcribed text here{]} or {[}transcribed text here{]} for italics.
\end{itemize}

You may notice gaps in the accession numbers, this is normal. Please continue to work with the numbers assigned in column A.

\end{document}
